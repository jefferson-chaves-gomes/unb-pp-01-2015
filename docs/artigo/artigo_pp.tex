\documentclass[12pt]{article}

\usepackage{sbc-template}
\usepackage{float}

\usepackage{graphicx,url}

%\usepackage[brazil]{babel}
\usepackage[latin1]{inputenc}

\makeatletter
\setlength{\@fptop}{0pt}
\makeatother

\sloppy

\title{Paralelismo em algoritmos de detec��o de adultera��es em imagens digitais}

\author{Vitor Filincowsky Ribeiro, Jefferson Chaves Gomes, Felipe Lopes de Souza}


\address{Instituto de Ci�ncias Exatas \\
Departamento de Ci�ncia da Computa��o - CIC \\
Universidade de Bras�lia (UnB) - Bras�lia, DF - Brasil
\email{vribeiro@cic.unb.br, \{jefferson.chaves, felipelopess\}@gmail.com}
}

\begin{document}

\maketitle

\section{Introdu��o}
\label{intro}

\section{Processamento de imagens digitais}

\section{Processamento paralelo}
\label{paral}

\subsection{Medidas de desempenho}
\label{perform}

O \textit{speed-up} � uma medida de desempenho que � dada pela raz�o entre os tempos de execu��o do programa serializado e do programa paralelizado, conforme equa��o \ref{eq-spdup}.

\begin{equation}
\label{eq-spdup}
S = \frac{t_{serial}}{t_{parallel}}
\end{equation}

A efici�ncia de um programa paralelizado � medida pela raz�o entre o \textit{speed-up} e a quantidade de n�cleos do processador, conforme equa��o \ref{eq-eff}.

\begin{equation}
\label{eq-eff}
E = \frac{S}{p} = \frac{t_{serial}}{p * t_{parallel}}
\end{equation}

\section{Aplica��o desenvolvida}
\label{desenv}

\section{An�lise dos resultados}
\label{result}

\section{Considera��es finais}
\label{final}


\end{document}
